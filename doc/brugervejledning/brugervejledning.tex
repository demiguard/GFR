\documentclass{article}
\usepackage[T1]{fontenc} %thanks's daleif
\usepackage[utf8]{inputenc}
\usepackage[danish]{babel}						% Dokumentets sprog


\author{Christoffer Vilstrup Jensen, Simon Surland Andersen}
\title{Brugervejledning}

\begin{document}
\tableofcontents
\section*{Introduktion}
Dette dokument beskriver dagligt brug af alle GFRcalc. Til udvikling og vedligholdelse af GFRcalc bør du referer dokumentet: "kode-dokumentation.pdf". Dette dokument er skrevet så det kan læses af alle bruger.\\\\
Brugervejledningen er skrevet til version 1.0 af GFRcalc.\\\\
Programmet er udarbejdet af Christoffer Vilstrup Jensen og Simon Surland Andersen. Den ansvarlige person for vedligeholdelse er Christoffer Vilstrup Jensen og kan kontakt via mail på Christoffer.Vilstrup.Jensen@regionh.dk
\section*{Ordbog}
\begin{itemize}
	\item[GFRcalc] - Dette program
	\item[GFR]     - Glomerular filtration rate
	\item[SP]      - Sundhedsplatformen
\end{itemize}
\section*{Formål}
GFRcalc er et værktøj som kan tilgås på hjemmesiden gfr/ eller gfr.petnet.rh.dk. værktøjet benyttes til udregning af Clearence undersøgelser. Mere specifik til at udregne GFR. 
\newpage
\section{Til dagligt brug}
For at kunne benytte nogle GFRcalc functioner, skal der logges ind.  
\subsection{Oversigt over nuværende undersøgelser}
Denne side præcenterer alle ikke afsluttede undersøgelser. Hvis en undersøgelse mangler kontakt din RIS/PACS ansvarlige person.\\
Gamle undersøgelser kan slettes, bemærk dog at undersøgelser ikke kan slettes, så længe at undersøgelsen er aktiv (Patienten er fremmødt, og undersøgelsen er ikke afsluttet.)\\
De forskellige tabler kan sortes efter: navn, cpr-nr, dato, procedure eller accession number, ved at klikke på toppen af den ønskede søjle.\\
En undersøgelse information kan tilgås via at klikke på den.\\\\
Man kan aflæse, hvor meget information, der findes inde i en undersøgelse via farven på 'clipboardet':
\begin{itemize}
	\item[Grøn] - Undersøgelsen har alle information og afvænter af blive godkendt og sendt til PACS
	\item[Gul] - Undersøgelsen har nogle bruger indtastet informationer, men har ikke udregnet clearance endnu. 
	\item[Rød] - Undersøgelse har ingen bruger indtastet informationer. 
\end{itemize} 
\textbf{Bemærk:} For nogle hospitaler kan der findes flere tabel ingang til samme undersøgelse. Det er vigtigt, at den rigtige undersøgelse udfyldes for at  SP kan generer, de korrekte reporter.    
\subsection{Udfyldning af undersøgelser} 
\subsection{Godkendelse og oversigt}
\subsection{Find gamle undersøgelse}
\subsection{Dokumentation}
Dokumentation beskriver metoden hvordan GFR værdierne er udregnet. 
\newpage
\section{Til administration}

\end{document}