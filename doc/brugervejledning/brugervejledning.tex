\documentclass{article}
\usepackage[T1]{fontenc}
\usepackage[utf8]{inputenc}
\usepackage[danish]{babel} % Set document language = danish

\setlength\parindent{0pt} % \noindent on the whole document

% URL styling - start
\usepackage{hyperref}
\hypersetup{
    colorlinks=true,
    linkcolor=black,%blue,
    filecolor=magenta,
    urlcolor=blue,%cyan,
}

\urlstyle{same}
% URL styling - end

\author{Christoffer Vilstrup Jensen, Simon Surland Andersen}
\title{Brugervejledning}

\begin{document}
\tableofcontents
\section*{Introduktion}
Brugervejledningen er skrevet til version 1.2.x af GFRcalc.\\

Dette dokument beskriver dagligt brug af GFRcalc. Dette dokument er skrevet så det kan læses af alle bruger. Til udvikling og vedligeholdelse af GFRcalc bør du referer til dokumentet: "code-documentation.pdf".\\

Programmet er udarbejdet af Christoffer Vilstrup Jensen og Simon Surland Andersen. Den ansvarlige person for vedligeholdelse er Christoffer Vilstrup Jensen og kan kontakt via mail på christoffer.vilstrup.jensen@regionh.dk\\\\
Alt data; \textit{Navne, cpr numre og test data}, i brugervejledningen er fabrikeret.

\section*{Ordbog}
\begin{itemize}
	\item[GFRcalc] 	- Dette program
	\item[GFR]     	- Glomerular Filtration Rate
	\item[SP]      	- Sundhedsplatformen
	\item[PACS]	   	- Picture Archiving and Communication Systems
	\item[RIS]		- Radiology Information System
\end{itemize}
\section*{Formål}
GFRcalc er et værktøj som kan tilgås på hjemmesiden \url{http://gfr01} eller \url{http://gfr.regionh.top.local}. værktøjet benyttes til udregning af Clearance undersøgelser. Mere specifik til at udregne GFR. 
\newpage
\section{Til dagligt brug}
For at kunne benytte nogle GFRcalc funktioner, skal der logges ind.
\subsection{Hurtig Guide}
Udfylding af undersøgelsen.
\begin{enumerate}
	\item Gå til hjemmesiden \url{http://gfr.regionh.top.local}.
	\item Log in, og navigerer til "Åben undersøgelse" fra sidebaren. 
	\item Vælg den relevante undersøgelse, ved at klikke på undersøgelsen.
	\item Udfyld undersøgelsen ved at indtaste relevant data.
	\begin{itemize}
		\item Brug knappen “Gem” til at gemme de nuværende indtastet data
	\end{itemize}
	\item Klik “Beregn”, hvorefter en graf vil blive genereret som beskriver undersøgelsen.
	\item Klik “Send til Kontrol” Hvis graffen er korrekt.
\end{enumerate} 
Kontrol af undersøgelsen
\begin{enumerate}
	\item Gå til hjemmesiden \url{http://gfr.regionh.top.local}.
	\item Log in, og navigerer til "Kontrol" fra sidebaren. 
	\item Vælg den relevante undersøgelse, ved at klikke på undersøgelsen.
	\item Godkend alle data felter og skriv dit bamID i det tomme felt
	\item Klik “Send til PACS” Hvis graffen er korrekt.
\end{enumerate} 

\subsection{Oversigt over nuværende undersøgelser (Åbn undersøgelse)}\label{list_study}
Denne side præsenterer alle undersøgelser, hvor patienten er ankomst registret. Dette betyder at hvis undersøgelsen bliver afsluttet i RIS, før at man har besøgt GFRcalc, så kan undersøgelsen ikke findes i GFRcalc. Hvis en undersøgelse mangler kontakt din RIS/PACS ansvarlige person. \\

Gamle undersøgelser kan slettes, bemærk dog at undersøgelser ikke kan slettes, så længe at undersøgelsen er aktiv (Patienten er fremmødt, og undersøgelsen er ikke afsluttet.) \\

Tabellen kan sorteres efter: navn, cpr-nr, dato, procedure eller accession number, ved at klikke på toppen af den ønskede søjle. \\

En undersøgelses information kan tilgås via at klikke på den. \\

Man kan aflæse, hvor meget information, der findes inde i en undersøgelse via farven på clipboardet (i søjlen med "status"):
\begin{itemize}
	\item[Grøn] - Undersøgelsen har alle information og afventer at blive godkendt og sendt til kontrol.
	\item[Gul] - Undersøgelsen har nogle bruger indtastet informationer, men har ikke udregnet clearance endnu.
	\item[Rød] - Undersøgelse har ingen bruger indtastet informationer.
\end{itemize}
\textbf{Bemærk:} For nogle hospitaler kan der findes flere tabel indgange til samme undersøgelse. Det er vigtigt, at den rigtige undersøgelse udfyldes for at  SP kan genere, de korrekte reporter.

\subsection{Udfyldning af undersøgelser}\label{fill_study}
Denne side tilgås ved at klikke på en af undersøgelserne fra “Åbn undersøgelse”\\\\
Øverst kan information om patienten og undersøgelsen indtastes. \\
Nederst i venstre hjørne kan man tilføje blodprøver og afslutte undersøgelsen. \\ 
Nederst i højre hjørne findes dagens tællinger fra “Wizarden”. Tællingerne er angivet efter tidspunkt og dato.  \\

Ekstreme værdier vil blive markeret med gul. Dette har ingen funktionel betydning ud over at gøre opmærksom på mulige taste fejl. Hvis en værdi  bliver indtastet i forkert format, eller værdien er umulig, vil den mærkeres med rød. GFRcalc kan kun udregn GFR, hvis der ikke findes nogle røde tal på siden. \\

Talende skal indtastes i den korrekte enhed. Disse enheder er:
\begin{itemize}
	\item Højde: Centimeter - (cm)
	\item Vægt: Kilogram - (kg)
	\item Sprøjtevægt: Gram - (g)
\end{itemize}
Ved alle decimaltal skal der benyttes: » , «\\%$\;$ i stedet for: » , «.\\
Alle datoer skal intaste i formatet DD-MM-ÅÅÅÅ.\\\\
For at tilføje en prøve eller tilføje en standard skal man vælge mindst 1 og op til 6 prøver fra Dagens tællinger. Hvis flere prøver vælges tages gennemsnittet af disse prøver. Hvis der er stor numerisk forskel på prøverne gives der en advarsel, som kan indikerer en mulig tastefejl.\\
Scanninger kan identificeres ved, tidspunktet hvor scanningen af første rack er færdig.\\

Det er vigtigt at brugeren selv husker hvilken patient, der findes i hvilken række og position.\\

Yderlige krav for at tilføje en prøve er, at udfylde hvilket tidspunkt prøven er taget på.\\ Datoerne for prøvetagningen sættes som standard til dagens dato.\\
Der kan maximalt tilføjes en prøve til en prøve modellerne. Der skal tilføjes mere end 1 prøve ved flere punkt modellen.\\

Prøver kan slettes ved at clike på det røde »X«. Prøver kan redigeres ved at klikke på låsen. I tilfældet hvor der er valgt forkerte tællinger, anbefales det at brugeren sletter den forkerte prøve og vælger de korrekte tællinger, i stedet for at brugeren redigere på den forkerte prøve.\\

Tællinger, der er mere end 24 timer gamle, bliver rykket til en backup. Tællinger fra backup kan hentes, i bunden af de nuværende tællinger. Her kan alle tællinger fra en dato hentes. De gamle tællingerne er identificeret på samme som nye tællinger.
\\

De indtastede informationer kan gemmes ved at klikke på 'Gem' knappen, som er placeret i nederst venstre hjørne. Hvis brugeren gemmer og et injektions tidspunkt ikke er udfyldt, så vil hjemmesiden selv udfylde 00:00.\\\\
Hvis man ønsker at beregne GFR, så skal man udfylde alle felter, samt et relevant antal prøver. Derefter bliver bruger videresendt til hjemmesiden beskrevet i section \ref{present_study}.\\

%\textbf{Bemærk:} Der kan opstå ventetid i forbindelse med lavet billede, da GFRcalc også skal fremskaffe alt historik om patients tidligere undersøgelser (fra Sommeren 2019 og frem). Denne historik inkluderer alle undersøgelser lavet med GFRcalc.  
\subsection{Beregn og oversigt}\label{present_study}
Denne side kan tilgås fra en udfyldt undersøgelse fra \ref{fill_study}.\\\\
Der præsenteres et billede med en graf og information indtaste med undersøgelsen. GFR-værdierne er genereret ud fra metoden beskrevet i \ref{Documentation}.  Nedenunder billedet findes 4 knapper, som kan bruges til følgende:
\begin{itemize}
	\item \textbf{Tilbage til hovedmenu} - Navigerer til hjemmesiden beskrevet i sektion \ref{list_study}
	\item \textbf{Tilbage til redigering} - Navigerer til hjemmesiden beskrevet sektion \ref{fill_study}
	\item \textbf{Billede til printning} - Navigerer til et fuldt skærmbillede af billede uden hjemmeside navigation, denne side er velegnet til printning af undersøgelsen.
	\item \textbf{Send til kontrol} - Sender undersøgelsen til sekundær kontrol før undersøgelsen sendes til PACS, hvor den kan tilgås fra "Kontrol"\ siden.
	%\item \textbf{Send til PACS} - Færdiggøre undersøgelsen og sender billede til PACS. Undersøgelsen bliver slettet fra GFRcalc, hvis GFRcalc succesfuldt sender billedet til PACS.\\ Det er ikke muligt at rette i undersøgelsen efter at den er sendt til PACS.\\
	%Hvis et billede med fejl bliver send til PACS, så skal du kontakt både en PACS ansvarlig og en GFRcalc admin. Oplyse dem om at der er sket en fejl.
	\item \textbf{QA-plot} \textit{Denne knap vises kun ved "flere prøve"\ metoden} - Navigerer til skærmbillede hvor man kan se et plot over hvor tæt prøveresultaterne ligger på den generetet linje.
\end{itemize}

\subsection{Kontrol}
Denne side findes i "Kontrol"\ i hovedmenuen og kan tilgås efter en undersøgelse er blivet "send til kontrol"\ fra beregn siden.\\

Efter en undersøgelse er blevet valgt, skal man godkende at alle felter er korrekt udfyldt, indtaste sit BamID og derefter sende undersøgelsen til PACS.\\

Hvis der findes fejl i undersøgelsen kan den sendes tilbage til redigering.

\subsection{Find gamle undersøgelse}\label{search}
Tidligerer afsluttet undersøgelser, som er lavet med GFRcalc kan findes med denne funktion.\\
Man kan søge på:
\begin{itemize}
	\item Dato
	\item Navn
	\item CPR number
	\item AccessionNummer
\end{itemize}
Efter at man klikker på søg, kan man sorterer søgeresultaterne efter søjlerne i tabellen ved at klikke på toppen af søjlen. Man kan se et søge resultat ved at klikke på resultatet.\\

\textbf{Bemærk} Søgetiden kan reduceres ved at man indskrænker sin søgning.

\subsection{Tilføj/Fjerne procedure filter}
Listen af undersøgelser på siden "Åbn undersøgelse"\ kan filtreres, så undersøgelser med bestemte "Procedure"\ navne ikke bliver vist. For eksempel hvis man ikke ønsker at se undersøgelser af Procedure typen "Clearance blodprøve 2. gang". For opsætning af sådanne filter gør følgende:
\begin{enumerate}
	\item Login med bruger som har administrator rettigheder.
	\item Klik på "Admin panel"\ i menuen til venstre.
	\item Vælg "Procedurer"\ på toppen af siden.
	\item Klik på "+"\ knappen.
	\item Skriv navnet på Proceduren man ikke ønsker at se længere, og klik "Tilføj".
	\item Gå tilbage til "Admin panel"\, vælg nu "Procedure filter"\ og klik på "+".
	\item Vælg afdelingen som filterede skal oprettes for, samt vælg id'et for det tidligere oprettede Procedure og klik "Tilføj".
\end{enumerate}

Filtre kan ligeledes fjernes ved at:
\begin{enumerate}
	\item Login med bruger som har administrator rettigheder.
	\item Klik på "Admin panel"\ i menuen til venstre.
	\item Vælg "Procedurer filter"\ på toppen af siden.
	\item Klik på det røde skraldespands ikon ud for det filter som ønskes slettet.
\end{enumerate}

\subsection{Metode Dokumentation}\label{Documentation}
Metode dokumentationen kan findes efter den tekniske Dokumentation



\section{Teknisk documentation}
Denne sektion er for tekniske side af GFR, så den har ikke noget information om daglig brug af GFRcalc.\\

GFRcalc laver dicom-objecter, med føglende tags, Tags som ikke er private føgler dicom standarden:

\begin{table}
\begin{center}
\begin{tabular}{|c|c|p{7cm}|}\hline
	Dicom Tag   & VR & Forklaring \\ \hline
	(0008,0018) & UI & genereret SOP Instance UID, genberegnes ved hver opdatering af Dicom objectet \\ \hline
	(0008,0020) & DA & Dato undersøgelse er booket til \\ \hline
	(0008,0021) & DA & Dato undersøgelse er booket til \\ \hline	
	(0008,0030) & TM & Tidspunkt undersøgelse er booket til \\ \hline
	(0008,0031) & TM & Tidspunkt undersøgelse er booket til \\ \hline	
	(0008,0050) & SH & Accession nummer af undersøgelsen \\ \hline 
	(0008,0060) & CS & Modalitet af undersøgelsen sat til OT \\ \hline
	(0008,0064) & CS & Billed type - værdi er konstant: SYN \\ \hline
	(0008,0080) & LO & Hospital som undersøgelsen er udført på, kan konfigures fra admin panelet \\ \hline
	(0008,0081) & LO & Addresse som undersøgelsen er udført på, kan konfigures fra admin panelet \\ \hline
	(0008,1010) & SH & AE-title som undersøgelsen er lavet ved \\ \hline
	(0008,1030) & LO & Undersøgelse beskrivelse, ej ændret men læst. \\ \hline
	(0008,103E) & LO & Beskrivelse af undersøgen og at metoden \\ \hline
	(0010,0010) & PN & Navn på patienten i dicom format \\ \hline
	(0010,0020) & LO & Patient CPR-nr. \\ \hline
	(0010,0030) & DA & Patient fødseldato på format YYYYMMDD \\ \hline
	(0010,1020) & DS & Patient højde i meter \\ \hline
	(0010,1030) & DS & Patient Vægt i Kg \\ \hline
	(0018,1019)	& LO & Beskrivelse af software og versions nummer \\ \hline
	(0020,000E) & UI & Genereret Series Instance UID, genberegnes ved vær opdatering af dicom objectet \\ \hline
	(0020,0010) & SH & ID til undersøgelse, i formattet GFR\# XXXXXXXX hvor X'erne er taget fra Accessionnummeret \\ \hline
	(0020,0011) & IS & Billed nummer, altid 1 \\ \hline
	(0020,0013) & IS & Billed nummer, altid 1 \\ \hline
	(0028,0002) & US & Antal byte per pixel, altid 3 \\ \hline
	(0028,0004) & CS & Kode for forståelse pixel data \\ \hline
	(0028,0010) & US & Billed højde i pixel, altid 1080 \\ \hline
	(0028,0011) & US & Billed brede i pixel, altid 1920 \\ \hline
	(0028,0100) & US & Antal bits allokeret per farvekanal \\ \hline
	(0028,0101) & US & Antal bits brugt per allokeret bits \\ \hline
	(0028,0102) & US & Størst betydende bit  \\ \hline
	(0028,0103) & US & Unsigned representation af farver\\ \hline
	(7FE0,0010) & OW & Ukomprimeret pixels til billedet af undersøgelsen.\\ \hline
\end{tabular}
\end{center}
\caption{Public tags manipuleret af GFRcalc}
\end{table}


\begin{table}[h]
	\begin{center}
		\begin{tabular}{|c|c|p{7cm}|}\hline
		Dicom Tag   & VR &  Forklaring \\ \hline
		(0023,0010) & LO &  Dicom allokering af tag gruppe (0023,10XX) \\ \hline
		(0023,1001) & LO & Konklusion af GFR undersøgelsen eg. Normal, moderat Nedsat, Svært nedsat \\ \hline 
		(0023,1002) & LO & Intern version nummer af Dicom object, ingen effekt på GFR værdien. \\ \hline
		(0023,1010) & LO & Metoden som er brugt til at regne GFR værdien: "En blodprøve, Voksen", "En blodprøve, Barn", "Flere blodprøver" \\ \hline
		(0023,1011) & LO & Metoden til at regne krops overflade, som er brugt i GFR. Haycock er den værdi, som er brugt\\ \hline
		(0023,1012) & DS & Clearance værdien regnet ud fra undersøgelsen, beskrevet i metode dokumentation\\ \hline
		(0023,1014) & DS & Clearance værdien normalizeret med krops overfladen til $1.73m^2$ \\ \hline
		(0023,1018) & DT & Tidspunktet for Injektion af radioaktivt materale i formattet YYYYMMDDHHMMSS \\ \hline
		(0023,1019) & US & Sprøjte nummer, ikke brugt til beregning af Clearance værdien \\ \hline
		(0023,101A) & DS & Udregnet vægt af injektion materiale, værdien er i gram \\ \hline
		(0023,101B) & DS & Sprøjte vægt før injektion, værdien er i gram  \\ \hline
		(0023,101C) & DS & Sprøjte vægt efter injektion, værdien er i gram  \\ \hline
		(0023,1020) & SQ & En liste hvor et element svare til en blodprøve \\ \hline
		(0023,1021) & DT & Liste element af (0023,1020): Tidspunkt hvor blodprøven er taget \\ \hline
		(0023,1022) & DS & Liste element af (0023,1020): Gennemsnit af Radioaktiv aktivitet i blodprøven \\\hline
		(0023,1023) & DS & Liste element af (0023,1020): Afvigelse udregnet via  maximal $M$ værdi fra gennemsnit og minimal værdi fra gennemsnittet $m$: $\frac{M-m}{M+m} \cdot 100$ \\ \hline
		(0023,1024) & DS & Standart tælletal af radioaktivt materiale \\ \hline
		(0023,1028) & DS & Fortynding faktor af radioaktivt materiale \\ \hline
		(0023,103F) & SQ & Liste af Historiske undersøgelser: En historisk undersøgelse har føglende tags: AccessionNumber, StudyDate, PatientSize, PatientWeight, Clearance Værdi (0023,1011), Normalized Clearance (0023,1012),  Injektions tidpunktet (0023,1018), Prøver fra undersøgelsen (0023,1020) \\ \hline
		(0023,1040) & LT & Kommentar til intern brug  \\ \hline
		\end{tabular}
	\end{center}
	\caption{Private tags manipuleret af GFRcalc}
\end{table}



\end{document}