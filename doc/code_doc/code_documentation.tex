\documentclass{article}
\usepackage[T1]{fontenc} %thanks's daleif
\usepackage[utf8]{inputenc}
\usepackage[danish]{babel}						% Dokumentets sprog
\usepackage{graphicx}							% contains the "includegraphics" command for inserting images

% Highlighted comments
\usepackage{color}
\newcommand{\makecomment}[1]{{\color{red} \bf Comment: #1}}

\title{Code Documentation}
\author{By: Christoffer Vilstrup Jensen \& Simon Surland Andersen}
\begin{document}
\maketitle

\section{Project overview}
\makecomment{Insert list of all files and directories and possibly a small description of each one.}

\section{Structure of stored data}


\section{Private DICOM tags}


\section{Formatting specifications}
\subsection{Commas and Dots (presenting floats)}


\subsection{Dates and timestamps}


\section{ae\_controller}


\section{API}
\makecomment{REST design}

\section{Database design}
\begin{figure}[h]
	\centering
	\includegraphics[scale=0.5]{images/gfr_database.png}
	\caption{Database relationship diagram}
\end{figure}

\section{Add a panel to admin panel / Database}

\subsubsection*{Introduction}
To enable database manipulation / editting without using tools like DB broswer and their like. There's a admin panel, that allows editting of DB entries. The following is a guide on how to edit the codebase s.t. it can display a new table

\begin{enumerate}
	\item Add the database schema by creating a model class in \textit{main\_page/models.py} file.
	\item Update the database by calling the django commands make migration and migrate.
	\item Create the class \textbf{Edit<model name>Form} corosponding to the edit form and the class \textbf{Add<Model name>Form} corosponding to the adding form in the \textit{main\_page/forms.py} file
	\item Add an Endpoint by adding the class \textbf{<model name>EndPoint} in the \textit{main\_page/views/api/api.py} file.
	\item Update \textit{main\_page/url.py} with the endpoint created in step 4, remember to create 2 endpoints for displaying and for accessing an object. 
	\item In the classes \textbf{AdminPanelEditView} and \textbf{AdminPanelAddview} found in the \textit{main\_page/views/admin\_panel.py} file, add an entry with to the dictionary \textbf{MODEL\_NAME\_MAPPINGS} with the model from step one in both classes. Add an Entry to the dictionary in the:\\ \textbf{EDIT\_FORM\_MAPPINGS} / \textbf{ADD\_FORM\_MAPPING} in the respective classes with the form you created in step 3.
	 \item Update the javascript in the files \textit{main\_page/static/main\_page/js/admin\_panel.js}, \textit{main\_page/static/main\_page/js/admin\_panel\_add.js} and\\ \textit{main\_page/static/main\_page/js/admin\_panel\_edit.js}
	 \item Add the option to the button in the admin
\end{enumerate} 
Total list of files that require to be edittede 
\begin{itemize}
	\item[] main\_page/
	\begin{itemize}
		\item[] forms/
		\begin{itemize}
			\item model\_add\_forms.py
			\item model\_edit\_forms.py
		\end{itemize}	
		\item models.py
		\item urls.py
		\item[] views/:
		\begin{itemize}
			\item admin\_panel.py
			\item[] api/
			\begin{itemize}
				\item api.py
			\end{itemize}
		\end{itemize}
		\item[] static/main\_page/js/
		\begin{itemize}
			\item admin\_panel.js
			\item admin\_panel\_add.js
			\item admin\_panel\_edit.js
		\end{itemize}
		\item[] template/main\_page/admin\_panel.html
	\end{itemize}
\end{itemize}
\end{document}